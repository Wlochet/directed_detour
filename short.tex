\documentclass[utf8,11pt]{article}
\usepackage[utf8]{inputenc}
\usepackage{fullpage}
\usepackage{authblk}
\usepackage{geometry}
\usepackage{amsthm,amsmath,amssymb}
\usepackage{xcolor}
\usepackage[pdftex, plainpages = false, pdfpagelabels, 
%pdfpagelayout = useoutlines,
bookmarks=false,
bookmarksopen = true,
bookmarksnumbered = true,
breaklinks = true,
linktocpage,
pagebackref,
colorlinks = true,  % was true
linkcolor = blue,
urlcolor  = blue,
citecolor = red,
anchorcolor = green,
hyperindex = true,
hyperfigures
]{hyperref} 
\usepackage[textwidth=1.1in]{todonotes}
\setlength{\marginparwidth}{2cm}
\makeatletter
\newcommand\tinyv{\@setfontsize\tinyv{4pt}{6}}
\makeatother
\usepackage{natbib}

\theoremstyle{plain}
\newtheorem{theorem}{Theorem}
\newtheorem{lemma}[theorem]{Lemma}
\newtheorem{corollary}[theorem]{Corollary}
\newtheorem{proposition}[theorem]{Proposition}
\newtheorem{fact}[theorem]{Fact}
\newtheorem{observation}[theorem]{Observation}
\newtheorem{claim}{Claim}[theorem]

\theoremstyle{definition}
\newtheorem{definition}[theorem]{Definition}
\newtheorem{example}[theorem]{Example}
\newtheorem{conjecture}[theorem]{Conjecture}
\newtheorem{open}[theorem]{Open Problem}
\newtheorem{problem}[theorem]{Problem}
\newtheorem{question}[theorem]{Question}
\newtheorem{assumption}{Assumption}
\title{On finding non-shortest paths in directed graphs}
\author{}
\date{}
\newcommand{\cst}{c}
\newcommand{\tanmay}[1]{\todo[size=\tinyv]{ TI: #1}}
\newcommand{\lr}[1]{\left(#1\right)}
\newcommand{\LR}[1]{\left\{#1\right\}}
\newcommand{\red}[1]{{\color{red}#1}}
\newcommand{\cP}{\mathcal{P}}
\newcommand{\cO}{\mathcal{O}}
\begin{document}
\maketitle 

We use $\odot$ for concatenation of paths. By simplifying an $st$-walk, we mean removing cycles to get an $st$-path. We define the level of a vertex by its distance from $s$.

\begin{theorem}\label{proof:main} 
   Shortest detour can be solved in polynomial time. 
\end{theorem}

Let $D = (V,A)$ be a directed graph with $n$ vertices and $m$ arcs, and $s,t \in V(D)$. We assume that there exists at least one $st$-path in $D$. The goal is to find an $st$-path which is not a shortest $st$-path. We denote by $d(u,v)$ the length of a shortest $uv$-path. A $uv$-\textit{detour} is a $uv$-path whose length is strictly greater than $d(u,v)$.

Let $L_i$ denote the set of vertices at distance $i$ from $s$ and let $V'$ denote the set of vertices that belong to a shortest $st$-path. We denote by $L'_i = L_i \cap V'$ and note that $L'_1, \dots, L'_{d(s,t)}$ is a partition of $V' \setminus \{s,t\}$. Let $D'$ denote the subgraph of $D$ induced by $V'$, where we add for every pair of vertices $uv$ of $D'$ an arc $uv$ if there exists a $uv$-path in $D$ whose internal vertices do not belong to $V'$. Let $A'$ denote the arc set of $D'$, we will associate a weight $w(uv)$ with every arc $uv$ in $A'$. 

The arcs of $D'$ can be partitioned into three sets: 
\begin{itemize}
    \item Arcs of the form $uv$ with $u \in L'_{i}$ and $v \in L'_{i+1}$ for some $i \in [d(s,t)-1]$. These arcs are called \textit{forward} arcs. The weight of these arcs is always 1.
    \item Arcs of the form $uv$ with $u \in L'_{i}$ and $v \in L'_{j}$ for some $i,j \in [d(s,t)]$ with $1 \leq j < i \leq d(s,t)$. These arcs are called \textit{backward} arcs. The weight $w(uv)$ of such an arc is defined as the length of the shortest $uv$-path in $D$ whose internal vertices do not belong to $V'$. 
    \item Arcs of the form $uv$ with $u \in L'_{j}$ and $v \in L'_{z}$ for $1 \leq j \leq z \leq d(s,t)$. These arcs are called \textit{crossing} arcs and the weight is also defined as the length of the shortest $uv$-path in $D$ whose internal vertices do not belong to $V'$.
\end{itemize}

\begin{lemma}
    Finding a shortest $st$-detour in $D$ is equivalent to finding a shortest $st$-detour in $D'$.
\end{lemma}

\begin{proof}
It is clear from the definition that shortest $st$-paths in $D'$ are exactly the shortest $st$-paths in $D$. Moreover, suppose $P$ is an $st$-detour in $D'$, then every arc of $P$ either corresponds to an arc of $D$ or a path between the two endpoints. Therefore, this corresponds to a walk $W$ in $D$ of the same length. Since vertices of $D'$ can only appear once in $W$, the $st$-path obtained by simplifying it is then an $st$-detour in $D$. Finally, any $st$-detour in $D$ can be mapped to an $st$-detour in $D'$ by replacing every subpath whose internal vertices do not belong to $V'$ by the corresponding arc. Therefore, finding a shortest $st$-detour in $D$ is equivalent to finding a shortest $st$-detour in $D'$. 
\end{proof}


\begin{lemma}
    Suppose $P$ is a shortest $st$-detour in $D'$. If $P$ contains a crossing arc, then it contains only one such arcs and no backward arcs.
\end{lemma}

\begin{proof}
    This immediately follows from the definition.
\end{proof}

Therefore, finding a shortest $st$-detour in $D'$ that uses crossing arcs can easily be done by taking the crossing arc $uv$ with $u \in L'_{j}$ and $v \in L'_{z}$ for $1 \leq j \leq d(s,t)$ that minimizes $w(uv) - (z-j)$. The rest of the proof focuses on finding a shortest $st$-detour in $D'$ that does not use crossing arcs, so we can assume that this set is empty.

Let $P$ be a shortest $st$-detour in $D'$, $P$ must then be using at least one backward arc. Let $u$ be the tail of the first backward arc of $P$ and denote by $z$ the integer such that $u \in L'_z$. Let $v$ be the second vertex of $P$ in $L'_z$. By definition, we know that $P[u,v]$ is entirely contained in $L'_1, \dots, L'_{z-1}$ and therefore by minimality, $P[v,t]$ is a shortest $vt$-path in $D'$. This motivates the following definition.

\begin{definition}
    Let $P$ be a $st$-detour in $D'$ and let $u$ and $v$ be two distinct vertices of $P$ in the same level $L'_z$ for some $z \in [t]$. We say that $(u,v)$ are the \textit{breaking points} of $P$ if $P[s,u]$ is a shortest $su$-path, $P[u,v]$ is entirely contained in $L'_1, \dots, L'_{z-1}$ and $P[v,t]$ is a shortest $vt$-path in $D'$.
\end{definition}

By the previous discussion, we know that any shortest $st$-detour $P$ in $D'$ must have two breaking points and the algorithm will proceed by enumerating all possible pairs.

Let us now \textbf{fix} $u$ and $v$ two vertices in $L'_z$ such that $(u,v)$ are the breaking points of the shortest detour $P$. Note that by definition, $P[v,t]$ can be replaced by any shortest $vt$-path in $D'$. Therefore, the problem reduces to finding two disjoint paths $P_1$, $P_2$, where $P_1$ is a shortest $su$-path and $P_2$ is a $uv$-path entirely contained in $L'_1, \dots, L'_{z-1}$. Note that, while $P_1$ only uses forward arcs, $P_2$ must use at least one backward arc, and possibly many. For every $uv$ path $P'$, if we denote by $xy$ the last backward arc of $P'$, then we will say that $P'[y,v]$ is the \textit{final part} of $P'$. In a way, when looking for $P_1$ and $P_2$, we do not really need $P_1$ and $P_2$ to be disjoint, but rather that $P_1$ and the \textbf{final part} of $P_2$ are disjoint. Indeed, suppose that $P'_1$ is a $su$-path using only forward arcs, and $P'_2$ is a $uv$-path whose final part is disjoint from $P'_1$ and only uses vertices in $L'_1, \dots, L'_{z-1}$ as internal vertices. Let $xy$ be the last backward arc of $P'_2$, then $P'_1$ and $P'_2[y,v]$ are disjoint. Now if $P'_1$ and $P'_2[u,y]$ intersect, we can extract from this closed walk a $sy$-path $P_{s,y}$ which is internally contained in $L'_1, \dots, L'_{z-1}$.
Therefore, if $P_3$ denotes any shortest $vt$-path in $D'$, then the union of $P_{s,y}$, $xy$, $P'_2[y,v]$ and $P_3$ is a $st$-path which uses at least one backward arc, and therefore it is an $st$-detour (TODO: figure). Note that this detour can have a different pair of breaking points, but this does not matter as we are enumerating over all possible pairs.

%Note that Abusing notation, all the detours mentionned in this section will concern detours with $(u,v)$ as breaking points. So in particular, when we say shortest detour, we mean shortest detour with $(u,v)$ as breaking points.

The algorithm will proceed in iterations, where at each iteration, it will maintain a set of vertices $O$ that can be used for the $su$-path $P_1$. For a set $O$, we define $I_O$ as the set of vertices $r$ of $L'_1, \dots, L'_{z-1}$ such that there exists two disjoint paths $P_{r,v}$ and $P_{s,u}$ using only forward arcs in $D'$ with the following property: 

\begin{itemize}
    \item $P_{r,v}$ is a $rv$-path such that $O \cap P_{r,v}$ is a subpath of $P_{r,v}$
    \item $P_{s,u}$ is a $su$-path using only vertices in $O$ as internal vertices
\end{itemize}

We say that $I_O$ is the set \textit{associated} to $O$ and the paths $P_{r,v}$ and $ P_{s,u}$ are said to \textit{certify} the membership of $r$ in $I_O$.
Because we are only using forward arcs, finding the set $I_O$ can always be computed via the standard disjoint pahts in DAGs algorithm (TODO, maybe expand for the subpath thing). We say that a set $O$ is \textit{safe} if $P_1$ is contained in $O$ and the final part of $P_2$ is contained in $I_O$ and intersects $O$ on a subpath. The following lemma follows easily from the definition. 

\begin{lemma}\label{lem:start_O}
   The set $O$ defined as the subset of vertices of $L'_1, \dots, L'_{z-1}$ that belong to a $su$-path in $D'$ is safe.
\end{lemma}

The algorithm will then start with the set $O$ defined as in Lemma \ref{lem:start_O} and its goal will be to remove elements of $O$, while ensuring it stays safe, until it finds an $st$-detour.

Let us start with the following easy remark:

\begin{proposition}\label{pro:IO}
   Let $O$ be a safe subset of vertices of $L'_1, \dots, L'_{z-1}$ and $x$ is an element of $O$ that can be reached from $u$ in $L'_1, \dots, L'_{z-1}$ via a path $H$ such that: 
   \begin{itemize}
    \item The internal vertices of $H$ are $I_O$-free
    \item The last arc of $H$ is a backward arc $yx$.
   \end{itemize}
   Then, denoting $P_{x,v}$ and $P_{s,u}$ the two paths certifying that $x \in I_O$, the $sv$-path obtained by simplifying $P_{s,u} \cdot H \cdot P_{x,v}$ is entirely contained in $(L'_1, \dots, L'_{z-1})$ and is longer than $d(s,v)$
\end{proposition}

At each iteration, the algorithm will start by computing the set of vertices $A_O$ that can be reached in $L'_1, \dots, L'_{z-1}$ from $u$ by paths avoiding $I_O$ for internal vertices and such that the last arc is a backward arc. This set can easily be computed with standard reachability algorithms. By proposition \ref{pro:IO}, if $A_O \cap O$ is non-empty, then we can easily find a $st$-detour and the algorithm stops. Suppose now that $O \cap A_O$ is empty and remember that the algorithm invariant states that the final part of $P_2$ is contained in $I_O$. What this means is that the path $P_2$ cannot completely avoid $I_O$ before its final part and the algorithm will try to use this information to remove elements of $O$. This is the main goal of the following lemma. 

\begin{lemma}
    There is a polynomial time algorithm which, given some safe set $O$, either finds an $st$-detour or outputs a subset $O' \subset O$ such that $O'$ is safe.
\end{lemma}

\begin{proof}
    Let $R_O$ denote the set of vertices of $\{L'_1, \dots, L'_{z-1} \} \cup \{u\}$ reachable in $D' \setminus I_O$ from $u$. Let $A_O$ denote the set of vertices reachable from $R_O$ by a backward arc. The algorithm starts by checking if there exist two disjoint paths, one from $s$ to $t$ and one from $A_0$ to $v$ using only forwards arcs in $D'$. If these paths exist, then we can easily obtain an $st$-detour as in the proof of Proposition \ref{pro:IO}(TODO: clean this), and this algorithm simply outputs this detour.



    \begin{assumption}\label{ass:no_path_pair}
        There are no two disjoint paths, one from $s$ to $t$ and one from $A_0$ to $v$, using only forward arcs in $D'$.
    \end{assumption}
    
    Under Assumption~\ref{ass:no_path_pair}, consider the first element $x$ of $P_2$ that does not belong to $R$. By definition of $R$, $x$ must then belong to $I_O$, and moreover, the arc $yx$ before $x$ on $P_2$ must be a forward arc. Consider, in $O$, the set $B_O$ of elements $y \in O$ such that there exists an $(A_O,y)$ forward path in $D'$ that is internally disjoint from $O$. It contains in particular the set $A_O \cap O$. We also let $C$ denote $I_O \cap O$. It is easy to see that the maximal forward path of $P_2$ that contains $x$ must contain a path from $B_O$ to $C$ in $O$ using only forward arcs. We will use this insight to find the elements to remove from $O$. Abusing notation, we will use $O$ to denote the DAG induced on the vertex set $O$ by the forward arcs. We say that a vertex $a$ is \textit{above} the vertex $b$ on $O$ if it belongs to a level $L'_j$ smaller than the one of $b$. 

    The algorithm will compute a sequence of vertices $x_1, \dots, x_r$ of increasing levels such that with the following properties: 
    \begin{itemize}
        \item For every $i \in [r]$ there exists a vertex $b_i$ from $B_0$ such that every $b_iu$ path in $O$ goes through $x_i$.
        \item For every $i \in [r]$, every $sx_i$-path in $O$ intersects all the $(B_O,C)$ paths of $O$ that use a vertex above $x_i$; moreover $x_i$ is not in $I_0$.  
        \item Either $x_r$ is adjacent to $u$; or the set $X_r$ of vertices $x$ below $x_r$ such that every $su$ path in $O$ containing $x$ must contain $x_r$ is non-empty and does not contain any element of $B_O$.   
    \end{itemize}


    \begin{claim}
        If this sequence exists, then $O' := O \setminus \{x_r\} \cup X_r$ is safe. 
    \end{claim}

    \begin{proof}   
   Assume first that $x_r \in P_1$. As explained before, we know that $P_2$ contains a $(B_O, C)$ path in $O$ which is disjoint from $P_1$. By the properties of $x_r$ and $X_r$ we have that the first element $x'$ of this path must be an element of $B_O$ which is not above $x_r$ and cannot belong to $X_r$. Let $P_x'$ be the path in $O$ from $x'$ to some vertex $y'$ in $C$ disjoint from $P_1$. This path is also disjoint from $X_r$. Indeed, since $x'$ is not above $x_r$ and does not belong to $X_r$, it means there exists a path from $s$ to $x'$ that does not contain $x_{r}$. Now if you could reach a vertex $y \in X_r$ from $x'$, this would mean there exists a path from $s$ to $X_r$ which avoids $x_r$, and this is not possible. This means that, denoting $P'_1$ the $st$ path obtained from $P_1$ by replacing the $x_rt$ subpath by a subpath in $X_r$, we still have that $P'_1$ and $P_x'$ are disjoint. 
   
   
   Now $y'$ belongs to $I_O$ which means that there exists a path $P_{y',v}$ in $D'$ using only forward arcs and such that $P_{y',v} \cap O$ is a subpath. Since $y' \in O$ this path is actually a prefix. Moreover, this prefix cannot intersect $X_r$ and we can conclude that $P_{y',v}$ is disjoint from $P'_1$. Therefore, the path $P_x' \cdot P_{y',v}$ is a $(B_O,v)$ path which is disjoint from $P'_1$, which is not possible by assumption. This means that $P_1$ cannot use $x_r$ and thus $X_r$.
   Now $y'$ belongs to $I_O$ which means that there exists a path $P_{y',v}$ in $D'$ using only forward arcs and such that $P_{y',v} \cap O$ is a subpath. Since $y' \in O$ this path is actually a prefix. Moreover, this prefix cannot intersect $X_r$ and we can conclude that $P_{y',v}$ is disjoint from $P'_1$. Therefore, the path $P_x' \cdot P_{y',v}$ is a $(B_O,v)$ path which is disjoint from $P'_1$, which contradicts Assumption~\ref{ass:no_path_pair}. This means that $P_1$ cannot use $x_r$ and thus $X_r$.

   In order to show that $O'$ is safe, we must now show that the final part of $P_2$ is contained in $I_{O'}$ and intersects $O'$ on a subpath. If this final part does not intersect $\{x_r\} \cup X_r$, then nothing needs to be done, so let us assume that it does and let $P'_2$ denote this intersection. First, since $P'_2$ is contained in $I_O$ and $x_r \not \in I_O$, we have that $x_r \not \in P'_2$. Moreover, the only arcs from $O \setminus X_r$ to $X_r$ have $x_r$ as tail. This means that, if $P'_2$ intersects $X_r$, then the first vertex of $P'_2$ must belong to $X_r$. However, since there is no arc from $O \setminus X_r$ to $X_r$ not using $x_r$, it means that $P'_2 \cap X_r$ is a prefix of $P'_2$ and therefore, the intersection of the final part of $P_2$ with $O'$ coresponds to $P'_2 \setminus X_r$, which is still a subpath of $P_2$. 
\end{proof}

Let us now show how the algorithm can construct the sequence of vertices $x_1, \dots, x_r$. Let $A_0$ be the set of maximal elements of $B_O$ on $O$ viewed as a partial order. In other words, every element of $B_O \setminus A_0$ is reachable from $A_0$ by a path in $O$ and no path exists in $O$ between two distinct elements of $A_0$. Let $R_v$ denote the set of vertices $r$ of $O$ such that there exists an $(r,v)$ forward path in $D'$ which is internally disjoint from $O$. 

\begin{claim}
    The maximum vertex flow in $O$ between $A_0$ and $\{u\} \cup R_v$ is at most $1$.
\end{claim}

\begin{proof}
Indeed, suppose that this is not the case and there exists two elements $a_1, a_2$ in $A_0$ as well as two disjoint paths $P_{a_1, u}$ and $P_{a_2,v'}$ such that: 
\begin{itemize}
    \item $P_{a_1, u}$ is a $(a_1,u)$-path 
    \item $P_{a_2,v'}$ is a $(a_2, v')$-path for some element $v' \in R_v$.
\end{itemize}

Then by definition of $A_0$, any path $P_{s,a_1}$ between $s$ and $a_1$ in $O$ must be disjoint from $P_{a_2,v'}$. Finally, if we let $P_{v',v}$ denote any $(v',v)$ forward path in $D'$ internally disjoint from $O$, we have that $P_{s, a_1} \cdot P_{a_1,u}$ and $P_{a_2,v'} \cdot P_{(v',v)}$ are disjoint which is a contradiction. 
Then by definition of $A_0$, any path $P_{s,a_1}$ between $s$ and $a_1$ in $O$ must be disjoint from $P_{a_2,v'}$. Finally, if we let $P_{v',v}$ denote any $(v',v)$ forward path in $D'$ internally disjoint from $O$, we have that $P_{s, a_1} \cdot P_{a_1,u}$ and $P_{a_2,v'} \cdot P_{(v',v)}$ are disjoint, which contradicts Assumption~\ref{ass:no_path_pair}. 
\end{proof}    

Therefore, by Menger's theorem, there exists a vertex separator $x_1$ between $A_0$ and $\{u\} \cup R_v$ in $O$. Let us now show that $x_1$ satisfies the first two properties required by the elements of the desired sequence $x_1, \dots, x_r$. The first property follows immediately from the fact that $x_1$ separates $A_0$ from $\{u\}$. 

Consider now some $(B_O, C)$-path $P$ in $O$ from some $b \in B_0$ above $x_1$ to some vertex $c \in I_0$. Since we know that there exists a path $P_c$ from $c$ to $R_v$ in $O$ and a path $P_b$ from $A_0$ to $b$, we get that $P_b \cdot P \cdot P_c$ is a $(A_0, R_v)$-path in $O$, which means that $x_1$ belongs to it. If the level of $c$ is lower than the level of $x_1$, then we know directly that $x_1 \in P$ which is what we wanted to show. In order, to conclude, let us assume that the level of $c$ is higher than the level of $x_1$ and remember that $P_b \cdot P$ is a $(A_0, c)$-path in $O$. As $c \in I_0$, consider $P_c$ and $P_{s,u}$ the certifying paths of $c$. Remember that $P_{v'} := P_c \cap O$ is a subpath, and since $c \in O$, it means that $P_{v'}$ is a path from $c$ to some vertex $v' \in R_v$. Now there are two cases. Either $P_b \cdot P$ is disjoint from $P_{s,u}$, but this means that $P_b \cdot P \cdot P_{v'}$ is a $(A_0, R_v)$ forward path disjoint from the $(u,v)$-path $P_{s,u}$, which we assumed is not possible. The other case is that $P_b \cdot P$ intersects $P_{s,u}$, but then this would contradict the fact that $x_1$ separates $A_0$ from $\{u\} \cup R_V$. That last case also implies that $x_1$ does not belong to $I_0$.
Consider now some $(B_O, C)$-path $P$ in $O$ from some $b \in B_0$ above $x_1$ to some vertex $c \in I_0$. Since we know that there exists a path $P_c$ from $c$ to $R_v$ in $O$ and a path $P_b$ from $A_0$ to $b$, we get that $P_b \cdot P \cdot P_c$ is a $(A_0, R_v)$-path in $O$, which means that $x_1$ belongs to it. If the level of $c$ is lower than the level of $x_1$, then we know directly that $x_1 \in P$ which is what we wanted to show. In order, to conclude, let us assume that the level of $c$ is higher than the level of $x_1$ and remember that $P_b \cdot P$ is a $(A_0, c)$-path in $O$. As $c \in I_0$, consider $P_c$ and $P_{s,u}$ the certifying paths of $c$. Remember that $P_{v'} := P_c \cap O$ is a subpath, and since $c \in O$, it means that $P_{v'}$ is a path from $c$ to some vertex $v' \in R_v$. Now there are two cases. Either $P_b \cdot P$ is disjoint from $P_{s,u}$, but this means that $P_b \cdot P \cdot P_{v'}$ is a $(A_0, R_v)$ forward path disjoint from the $(s,u)$-path $P_{s,u}$, which contradicts Assumption~\ref{ass:no_path_pair}. The other case is that $P_b \cdot P$ intersects $P_{s,u}$, but then this would contradict the fact that $x_1$ separates $A_0$ from $\{u\} \cup R_v$. That last case also implies that $x_1$ does not belong to $I_0$.


So now that we have shown how the algorithm can construct $x_1$, let us assume that it has constructed a sequence of vertices $x_1, \dots, x_j$ of increasing levels such that for all of $ i\in [j]$, the first two properties are satisfied. In particular: 
    \begin{itemize}
        \item There exists a vertex $b_{j}$ in $B_O$ such that every $b_{j}u$-path in $O$ goes through $x_{j}$.
        \item Every $sx_j$-path in $O$ intersects all the $(B_0,C)$ paths of $O$ that use a vertex above $x_j$; moreover $x_j$ is not in $I_0$.  
    \end{itemize}

Let us now show how to construct $x_{j+1}$. Here there are basically two cases. The first one is if $X_r$ of vertices $x$ below $x_r$ such that every $st$ path in $O$ containing $x$ must contain $x_r$ is non-empty. Note that such a set, if it exsists can easily be found. If $X_j$ does not contain any element of $B_0$, then the sequence  $x_1, \dots, x_j$ with $r := j$ actually satisfies all the desired properties, and we are done. So let us assume that this is not the case and denote by $A_j$ the set of maximal elements of $B_0$ inside $X_{j}$. Again, we can show that the maximal flow between $A_{j}$ and $\{u\} \cup R_v$ is at most one and there exists a vertex $x_{j+1}$ separating $A_j$ from $\{u\} \cup R_v$. Using the exact same argument as for $x_1$, we can show that: 

\begin{itemize} 
    \item There exists a vertex $b_{j+1}$ in $B_O$ such that every $b_{j+1}u$-path in $O$ goes through $x_{j+1}$. 
    \item For every $i \in [r]$, every $sx_{j+1}$-path in $O$ intersects all the $(B_0,C)$ path of $O$ that use a vertex of $X_j$ above $x_{j+1}$; moreover $x_{j+1}$ is not in $I_0$.
\end{itemize}

Moreover, since we know that every $sx_{j+1}$ path must go through $x_{j}$, we have that every $sx_{j+1}$-path in $O$ intersects all the $(B_O,C)$ paths of $O$ that use a vertex above $x_{j}$, or $x_{j}$ himself. To conclude that case, we have to consider only $(B_O,C)$-paths that uses a vertex above $x_{j+1}$, disjoint from $X_{j} \cup \{x_j \}$ but not above $x_j$. Let $P$ denote such path with some $c \in I_O$ as endpoint. Moreover, assume that there exists a path $P_j$ from $s$ to $x_{j+1}$ which is disjoint from $P$. By definition of $I_0$, there exists a path $P_c$ from $c$ to $v$ such that the intersection $P'_c$ with $O$ is a prefix of $P$. Because $c$ is not above $x_j$ and is not in $X_j$, it means that there exists a path from $s$ to $c$ that is disjoint from $x_j$. Therefore, if $P'_c$ intersects $X_j$, it means that we can find a path in $O$ from $s$ to $X_j$ which is $x_j$ free, and this is not possible. So $P'_c$ is disjoint from $X_j$ and thus, 
if we denote by $P'_j$ the path obtained from $P_j$ by extending it into an $su$-path using vertices in $X_j$, we get that $P'_j$ is a $su$-path disjoint from the $(A_0,v)$ path $P \cdot P_c$, which contradicts Assumption~\ref{ass:no_path_pair}.

Finally, let us assume that the set $X_j$ is empty and consider $O_j$ the subset of vertices that belongs to a $su$-path in $O$ disjoint from $x_j$. Let $T_j := O \setminus O_j$ and note that elements of $T_j$ are exactly the elements $y$ such that every $sy$ paths in $O$ must contain $x_j$. Let $B_j$ the set of elements of $O_j$ containing $O_j \cap B_O$ and the set of elements of $O_j$ such that there exists a path from $x_j$ or $B_O \cap T_j$ with internals vertices in $T_j$. Let $A_j$ now denote the set of maximal elements of $B_j$ when considering $O_j$ as a partial order. The idea here will to use the same argument as for $x_1$, using $A_j$ instead of $A_1$. Remember, that there exists a vertex $b_j \in B_O$ such that 

\begin{claim}
    The maximum vertex flow between $A_j$ and $\{u\} \cup R_v$ in $O_j$ is at most one
\end{claim}

\begin{proof}
Indeed, suppose that this is not the case and there exists two elements $a_1, a_2$ in $A_j$ as well as two disjoint paths $P_{a_1, u}$ and $P_{a_2,v'}$ such that: 
\begin{itemize}
    \item $P_{a_1, u}$ is a $(a_1,u)$-path 
    \item $P_{a_2,v'}$ is a $(a_2, v')$-path for some element $v' \in R_v$.
\end{itemize}

Then by definition of $A_j$, any path $P_{s,a_1}$ between $s$ and $a_1$ in $O$ must be disjoint from $P_{a_2,v'}$. Finally, if we let $P_{v',v}$ denote any $(v',v)$ forward path in $D'$ internally disjoint from $O$, we have that $P_{s, a_1} \cdot P_{a_1,u}$ and $P_{a_2,v'} \cdot P_{(v',v)}$ are disjoint. If $a_2 \in B_O$, we get a contradiction with Assumption~\ref{ass:no_path_pair}.
If $a_2 \not \in B_O$, then there exists a path in $T_j$ from $B_O$ or $x_{j+1}$ 
with internals vertices. In either case, we can extend $P_{a_2,v'} \cdot P_{(v',v)}$ into a path from $B_O$ to $v$ which is disjoint from $P_{s, a_1} \cdot P_{a_1,u}$ as this latter path is in $O_j$. 
\end{proof}    
Let $x_{j+1}$ denote the vertex separator between $A_j$ and $\{u\} \cup R_v$ in $O_j$. We claim again that $x_1, \dots, x_j, x_{j+1}$ satisfy the desired properties, which will end the proof of the lemma as computing $O_j$ and $A_j$ can easily be done in polynomial time, and thus find $x_{j+1}$ as well. 

The first property is immediate. Indeed, if some element $b_{j+1} \in B_0 \cap A_j$, then it immediately satisfies said property by definition of $x_{j+1}$. Likewise, if there exists an element $x \in B_O \cap T_j$ with a path from $x$ to $A_{j}$ in $T_j$, then $x$ also satisfies this property. And finally, if all the elements of $A_j$ corresponds to extremeties of $(x_{j}, O_j)$-paths with internal vertices in $T_j$, then $b_j$ satisfies this property. 

Let us now show that 



\end{proof}










\end{document}
then there exists a $su$-path $P_{s,u}$ and a $xv$-path $P_{x,v}$ using only forward arcs in $D'$ that are disjoint. Therefore, the union of $P_{s,u}$, $P_2[u,y]$, $yx$, and $P_{x,v}$ is a $st$-detour. The second task of the algorithm is to update the set $F$ by removing vertices while maintaining the following invariant:

, initialized as the union of $L'_1, \dots, L'_{z}$. It will also compute the set of vertices $I_{R} \subseteq V'$ such that for every $r \in I_R$, we can find two disjoint paths in $D'$: a $rv$-path $P_{r,v}$ and a $su$-path $P_{s,u} \subseteq R$ that uses forward arcs in $D'$. Intuitively, $I_R$ corresponds to the set of vertices that could potentially be in the last part of $P_2$. Note that because we are only using forward arcs, finding such paths can be done in polynomial time using the standard disjoint paths in DAGs algorithm. We will say that $I_R$ is the set \textit{associated} to $R$. Note that, using the previous notation, all the vertices of $P_{r,v}$ must be contained in $I_R$. Indeed, if $x \in P_{r,v}$, then $P_{r,v}[r,x]$ and $P_{s,u}$ are two disjoint paths using only forward arcs.

In each iteration, the algorithm will have two separates task. The first one is to find the pair of paths $P'_1$ and $P'_2$ with the following properties: 
\begin{itemize}
    \item $P'_1$ is a shortest $su$-path using only vertices of $R$ and forward arcs of $D'$.
    \item $P'_2$ is a $uv$-path whose final part is disjoint from $P'_1$ and only uses vertices in $I_R \cup \{v\}$. Moreover, $P'_2$ does not use any vertex of $R_V$ before its last backward arc.
    \item The length of $P'_2$ is minimized.
\end{itemize}
We call any pair of paths respecting the first two properties a \textit{valid pair} of paths for $V$. Note that by the previous discussion, having a valid pair gives rises to a $st$-detour of length at most $|P'_1| + |P'_2| + d(v,t)$. The following lemma basically follows from doing a shortest path algorithm in $(L'_1 \cup, L'_2, \dots, L'_{z} )\setminus I_R$. 

\begin{lemma}\label{lem:valid_pair}
    Given $R$ and its associated set $I_R$, finding a valid pair of paths $P'_1$ and $P'_2$ for $R$ that minimizes $|P'_2|$ can be done in polynomial time. 
\end{lemma}
The second task of the algorithm is to update the sets $V$ and remove vertices that cannot be used in $P_1$. This is the hard part of the algorithm and the goal will be to maintain the following invariant: 

\textbf{Invariant:} If the algorithm has not yet found a detour of length at most the shortest $st$-detour with $(u,v)$ as breaking points, then there exists a shortest $st$-detour $P$ with $(u,v)$ as breaking points such that $P_1$ is entirely contained in $R$.

To maintain the invariant throughout the algorithm, we will use the following lemma.

\begin{lemma}\label{lem:update_R}
    There exists an algorithm running in polynomial time such that the following properties are satisfied.
    Suppose that $R$ is a set of vertices such that some shortest $st$-detour $P$ with $(u,v)$ as breaking points has its final part of $P_2$ entirely contained in $R$. Moreover, suppose that no valid pair for $R$ gives rises to a shortest $st$-detour. Then the algorithm returns a set $R' \subset R$ such that $P_1$ is entirely contained in $R'$.
\end{lemma}

Using the algorithm of the previous lemmas, the overall algorithm for a pair $(u,v)$ works as follows. We initialize $R$ as the union of $L'_1, \dots, L'_{z}$ and repeat the following two steps until $R$ is empty.
\begin{itemize}
    \item Use the algorithm of Lemma \ref{lem:valid_pair} to find a valid pair of paths $P'_1$ and $P'_2$ for $R$ that minimizes $|P'_2|$.
    \item Update $R$ using the algorithm of Lemma \ref{lem:update_R}.
\end{itemize}
Finally, the algorithm returns the minimum length detour found during the execution.


\section*{Proof of Lemma \ref{lem:update_R}}
Let us now turn our attention to the proof of Lemma \ref{lem:update_R}. Suppose now we are given a set $R$ satisfying the condition of the lemma, meaning that there exists some shortest $st$-detour $P$ with $(u,v)$ as breaking points such that $P_1$ is entirely contained in $R$. Let $I_R$ be the set associated to $R$ and $B := \{u\} \cup ( L'_1 \cup L'_2 \cup \dots \cup L'_{z-1} )\setminus I_R$. Let us denote by $A$ the vertices of $B$ reachable from $u$ in $D'[B]$. Consider now $P_2$ the $uv$-path of $P$. By definition of $I_R$, the final part of $P_2$ must be contained in $I_R$. Let $x$ be the first vertex of $P_2 \cap I_R$ and let $y$ be the predecessor of $x$ in $P_2$. Since $u \in A$ and by definition of $A$, we have that $y \in A$. We will distinghuish two cases depending on whether $yx$ is a backward arc or a forward arc.

\begin{claim}
If $yx$ is a backward arc, then there exists a valid pair for $R$ that gives rises to a shortest $st$-detour.
\end{claim}

\begin{proof}
    Indeed, by definition of $R$, there exists two disjoint paths $P_{s,u}$ and $P_{x,v}$ in $D'$ that only uses forwards arcs. Moreover, we have that $P_{s,u}$ is entirely contained in $R$ and $P_{x,v}$ is entirely contained in $I_R$. Therefore, the pair $(P_{s,u}, P_2[u,y] \odot yx \odot P_{x,v})$ is a valid pair for $R$ that gives rises to a shortest $st$-detour. Indeed, $|P_{s,u}| = P_1$, and $|P_{x,v}| \leq P_2[x,v]$ by definition of $P_{x,v}$. Therefore, $|P_{s,u}| + |P_2[u,y]| + 1 + |P_{x,v}| \leq |P_1| + |P_2|$.
\end{proof}

Using the previous claim and the assumption that no valid pair for $R$ gives rise to a shortest $st$-detour, we deduce that $yx$ is a \textbf{forward} arc. Remember that we can always assume that every vertex of $R$ belongs to a $st$ path using only forward arcs. Let $X_A$ denote the set of vertices of $A$ that are the endpoints of a backward arc starting from $A$. By definition, we then have that $P_2$ contains a forward path in $D'$ from $X_A$ to $y$. The main goal of the lemma is then to understand the structures of $(X_A, I_R)$ forward paths in $D'$ to remove vertices from $R$ while maintaining the invariant. More precisely, how these $(X_A, I_R)$ forward paths can intersect the $su$ forward paths in $D'[R]$. Since we are only interested in forward paths, we can simply work with the DAG $D''$ obtained from $D'[R]$ by removing all backward arcs. Let $F_A$ denote the set of vertices of $R$ reachable from $X_A$ in $D'$ by a forward path which is internally disjoint from $R$. In particular, $F_A$ contains $X_A \cap R$. Let $I'_R$ denote the set of vertices of $R$ that can reach $I_R$ in $D'$ by a forward path which is internally disjoint from $R$. Note that $I'_R$ contains $I_R \cap R$ and by definition of $I_R$, any pair of $(s,u)$ path and $(F_A,I'_R)$ path in $D''$ must intersect. 


\begin{claim}\label{claim:grid}
    Noting $s := s_0$, there exists an integer $l \geq 1$ and a sequence of vertices $s_0, s_1, \dots, s_l$ of $D''$ such that for every $i \in [l-1]$, there is a path from $s_i$ to $s_{i+1}$ in $D''$. Moreover, denoting $D_i$ the set of vertices contained on such paths, we have that: 
    \begin{enumerate}
        \item For every $i \in [l-1]$, $D_i \cap F_A \neq \emptyset$ and $F_A \subseteq D_1 \cup D_1 \cup \dots \cup D_{l}$
        \item For every $i \in [l-1]$, $s_i$ is a vertex cup between $s$ and $F_A \setminus (D_1, \cup D_2, \dots, \cup D_{i-1})$
        \item For every $i \in [l-1]$, all $(D_i\cap F_A,I'_R)$-paths intersect $s_i$.
    \end{enumerate}
    Additionally, such a sequence can be found in polynomial time.
\end{claim}

\begin{proof}
    Consider $F_{A_0}$ the set of vertices of $F_A$ that are reachable from $s$ in $D''$ by a path which is internally disjoint from $F_A$. Since every $su$ path in $D''$ intersects every $(F_A,I'_R)$-path, we have that the maximum flow in $D''$ between $F_{A_0}$ and $\{u\} \cup I'_R$ is $1$. Indeed, suppose there exists a path $H_1$ from $x \in F_{A_0}$ to $u$ and a path $H_2$ from $y \in F_{A_0}$ to $I'_R$ that are disjoint. Since $x \in F_{A_0}$, there exists a path $H$ from $s$ to $x$ in $D''$ which is  disjoint from $y$ and therefore $H \odot H_1$ is a $(s,u)$-path in $D''$ which is disjoint from the $(F_A,I'_R)$ path $H_1$. Therefore, by Menger's theorem, there exists a vertex $s_1$ that separates $F_{A_0}$ from $\{u\} \cup I'_R$. Let $D_1$ be the set of vertices on a path from $s$ to $s_1$ in $D''$, we will show that $D_1$ and $s_1$ satisfy the last two properties of the claim. Suppose first that $a \in F_A \setminus D_1$ and consider a path $H$ from $s$ to $a$. By definition of $F_{A_0}$, we know that some vertex $x \in F_{A_0}$ must belong to that path. Moreover, by definition of $D''$, this path can be extended into a $su$ path, which means that $s_1$ must belong to that path, and since $a \not \in D_1$, it must appear before $s_1$ on that path. Therefore, $s_1$ is a vertex cut between $s$ and $F_A \setminus D_1$. Finally, the fact that $D_1$ is disjoint from $I'_R$ and every $(D_1,I'_R)$-path intersects $s_1$ follows from the definition of $s_1$.


    Assume now that the vertices $s_0, s_1, \dots s_j$ have been constructed for some $h \geq 1$ such that all the last two properties of the lemma hold for every $i \in [j]$ but $F_A \setminus (D_1, \cup D_2, \cdot, \cup D_{j})$ is not empty.  We will show how to construct $s_{j+1}$ in a similar way to $s_1$. Consider $F_{A_j}$ the set of vertices of $F_A$ that are reachable from $s_j$ in $D''$ by a path which is internally disjoint from $F_A$. Again, the maximum flow in $D''$ between $F_{A_j}$ and $\{u\} \cup I'_R$ is $1$. Indeed, suppose there exists a path $H_1$ from $x \in F_{A_j}$ to $u$ and a path $H_2$ from $y \in F_{A_j}$ to $I'_R$ that are disjoint. Since $x \in F_{A_j}$, there exists a path $H$ from $s_j$ to $x$ in $D''$ which is  disjoint from $y$ and therefore $H \odot H_2$ is a $(s_j,v)$-path in $D''$ which is disjoint from the $(F_A,v)$ path $H_1$. Moreover, since $D''$ is a DAG and $y$ is reachable from $s_j$, any $ss_j$ path is $D''$ must also be disjoint from $H_1$, which contradicts the fact that every $su$ path in $D''$ intersects every $(F_A,I'_R)$-path. Therefore, by Menger's theorem, there exists a vertex $s_{j+1}$ that separates $F_{A_j}$ from $\{u\} \cup I'_R$. Let $D_{j+1}$ be the set of vertices on a path from $s_j$ to $s_{j+1}$ in $D''$, we will show that $D_{j+1}$ and $s_{j+1}$ satisfy the last two properties of the claim. Again, the last property directly follows from the fact that $s_{j+1}$ is a $(F, I'_R)$ Suppose first that $a \in F_A \setminus (D_1, \cup D_2, \cup D_{j} \cup D_{j+1})$ and consider a path $H$ from $s$ to $a$. By definition of $F_{A_j}$, we know that some vertex $x \in F_{A_j}$ must belong to that path. Moreover, by definition of $D''$, this path can be extended into a $su$ path, which means that $s_{j+1}$ must belong to that path, and since $a \not \in (D_1, \cup D_2, \cup D_{j} \cup D_{j+1})$, it must appear before $s_{j+1}$ on that path. Therefore, $s_{j+1}$ is a vertex cut between $s$ and $F_A \setminus (D_1, \cup D_2, \cup D_{j})$. Finally, the fact that $D_{j+1}$ is disjoint from $I'_R$ and every $(D_{j+1},I'_R)$-path intersects $s_{j+1}$ follows from the definition of $s_{j+1}$.
\end{proof}
