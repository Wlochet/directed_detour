\documentclass[utf8,11pt]{article}
\usepackage[utf8]{inputenc}
\usepackage{fullpage}
\usepackage{authblk}
\usepackage{geometry}
\usepackage{amsthm,amsmath,amssymb}
\usepackage{xcolor}
\usepackage[pdftex, plainpages = false, pdfpagelabels, 
%pdfpagelayout = useoutlines,
bookmarks=false,
bookmarksopen = true,
bookmarksnumbered = true,
breaklinks = true,
linktocpage,
pagebackref,
colorlinks = true,  % was true
linkcolor = blue,
urlcolor  = blue,
citecolor = red,
anchorcolor = green,
hyperindex = true,
hyperfigures
]{hyperref} 
\usepackage[textwidth=1.1in]{todonotes}
\setlength{\marginparwidth}{2cm}
\makeatletter
\newcommand\tinyv{\@setfontsize\tinyv{4pt}{6}}
\makeatother
\usepackage{natbib}

\theoremstyle{plain}
\newtheorem{theorem}{Theorem}
\newtheorem{lemma}[theorem]{Lemma}
\newtheorem{corollary}[theorem]{Corollary}
\newtheorem{proposition}[theorem]{Proposition}
\newtheorem{fact}[theorem]{Fact}
\newtheorem{observation}[theorem]{Observation}
\newtheorem{claim}{Claim}[theorem]

\theoremstyle{definition}
\newtheorem{definition}[theorem]{Definition}
\newtheorem{example}[theorem]{Example}
\newtheorem{conjecture}[theorem]{Conjecture}
\newtheorem{open}[theorem]{Open Problem}
\newtheorem{problem}[theorem]{Problem}
\newtheorem{question}[theorem]{Question}
\title{On finding non-shortests path in directed graphs}
\author{}
\date{}
\newcommand{\cst}{c}
\newcommand{\tanmay}[1]{\todo[size=\tinyv]{ TI: #1}}
\newcommand{\lr}[1]{\left(#1\right)}
\newcommand{\LR}[1]{\left\{#1\right\}}
\newcommand{\red}[1]{{\color{red}#1}}
\newcommand{\cP}{\mathcal{P}}
\newcommand{\cO}{\mathcal{O}}
\begin{document}
\maketitle 

We use $\odot$ for concatenation of paths.

\begin{theorem}\label{proof:main} 
   Shortest detour can be solved in polynomial time. 
\end{theorem}

Let $D = (V,A)$ be a directed graph with $n$ vertices and $m$ arcs, and $s,t \in V(D)$. We assume that there exists at least one $st$-path in $D$. The goal is to find an $st$-path which is not a shortest $st$-path. We denote by $d(u,v)$ the length of a shortest $uv$-path. A $uv$-\textit{detour} is a $uv$-path whose length is strictly greater than $d(u,v)$.

Let $L_i$ denote the set of vertices at distance $i$ from $s$ and let $V'$ denote the set of vertices that belong to a shortest $st$-path. We denote by $L'_i = L_i \cap V'$ and note that $L'_1, \dots, L'_t$ is a partition of $V' \setminus \{s,t\}$ where $t = d(s,v)$. Let $D'$ denote the subgraph of $D$ induced by $V'$, where we add for every pair of vertices $uv$ of $D'$ an arc $uv$ if there exists a $uv$-path in $D$ whose internal vertices do not belong to $V'$. Let $A'$ denote the arc set of $D'$, we will associate a weight $w(uv)$ with every arc $uv$ in $A'$. 

The arcs of $D'$ can be partitioned into three sets: 
\begin{itemize}
    \item Arcs of the form $uv \in$ with $u \in L'_{i}$ and $v \in L'_{i+1}$ for some $i \in [t-1]$. These arcs are called \textit{forward} arcs. The weight of these arcs is always 1.
    \item Arcs of the form $uv$ with $u \in L'_{i}$ and $v \in L'_{j}$ for some $i,j \in [t]$ with $1 \leq j < i \leq t$. These arcs are called \textit{backward} arcs. The weight $w(uv)$ of such an arc is defined as the shortest $uv$-path in $D$ whose internal vertices do not belong to $V'$. 
    \item Arcs of the form $uv$ with $u \in L'_{j}$ and $v \in L'_{z}$ for $1 \leq j \leq z \leq t$. These arcs are called \textit{crossing} arcs and the weight is also defined as the length of the shortest $uv$-path in $D$ whose internal vertices do not belong to $V'$.
\end{itemize}

\begin{lemma}
    Finding a shortest $st$-detour in $D$ is equivalent to finding a shortest $st$-detour in $D'$.
\end{lemma}

\begin{proof}
It is clear from the definition that shortest $st$-paths in $D'$ are exactly the shortest $st$-paths in $D$. Moreover, suppose $P$ is an $st$-detour in $D'$, then every arc of $P$ either corresponds to an arc of $D$ or a path between the two endpoints. Therefore, this corresponds to a walk $W$ in $D$ of the same length. Since vertices of $D'$ can only appear once in $W$, the $st$-path obtained by simplifying it is then an $st$-detour in $D$. Finally, any $st$-detour in $D$ can be mapped to an $st$-detour in $D'$ by replacing every subpath whose internal vertices do not belong to $V'$ by the corresponding arc. Therefore, finding a shortest $st$-detour in $D$ is equivalent to finding a shortest $st$-detour in $D'$. 
\end{proof}


\begin{lemma}
    Suppose $P$ is a shortest $st$-detour in $D'$. If $P$ contains a crossing arc, then it contains only one such arcs and no backward arcs.
\end{lemma}

\begin{proof}
    This immediately follows from the definition.
\end{proof}

Therefore, finding a shortest $st$-detour in $D'$ that uses crossing arcs can easily be done by taking the crossing arc $uv$ with $u \in L'_{j}$ and $v \in L'_{z}$ for $1 \leq j \leq t$ that minimizes $w(uv) - (z-j)$. The rest of the proof focuses on finding a shortest $st$-detour in $D'$ that does not use crossing arcs, so we might assume that this set is empty.

Let $P$ be a shortest $st$-detour in $D'$, $P$ must then be using at least one backward arc. Let $u$ be the tail of the first backward arc of $P$ and denote by $z$ the integer such that $u \in L'_z$. Let $v$ be the second vertex of $P$ in $L'_z$. By definition, we know that $P[u,v]$ is entirely contained in $L'_1, \dots, L'_{z-1}$ and therefore by minimality, $P[v,t]$ is a shortest $vt$-path in $D'$. This motivates the following definition.

\begin{definition}
    Let $P$ be a $st$-detour in $D'$ and let $u$ and $v$ be two distinct vertices of $P$ in the same level $L'_z$ for some $z \in [t]$. We say that $(u,v)$ are the \textit{breaking points} of $P$ if $P[s,u]$ is a shortest $su$-path, $P[u,v]$ is entirely contained in $L'_1, \dots, L'_{z-1}$ and $P[v,t]$ is a shortest $vt$-path in $D'$.
\end{definition}

By the previous discussion, we know that any shortest $st$-detour $P$ in $D'$ must have two breaking points and the algorithm will work by enumerating all possible pairs.

Let us now fix $u$ and $v$ two vertices in $L'_z$ and let us assume $(u,v)$ are the breaking points of the shortest detour $P$. Note that by definition, $P[v,t]$ can be replaced by any shortest $vt$-path in $D'$. Therefore, the problem reduces to finding two disjoint paths $P_1$, $P_2$, where $P_1$ is a shortest $su$-path and $P_2$ is a $uv$-path entirely contained in $L'_1, \dots, L'_{z-1}$. Note that, while $P_1$ only uses forward arcs, $P_2$ must use at least one backward arcs, and possibly many. For every $uv$ path $P'$, if we denote by $xy$ the last backward arc of $P'$, then we will say that $P'[y,v]$ is the \textit{final part} of $P'$. In a way, when looking for $P_1$ and $P_2$, we do not really need $P_1$ and $P_2$ to be disjoint, but rather that $P_1$ and the \textbf{final part} of $P_2$ are disjoint. Indeed, suppose that $P'_1$ is a $su$-path using only forward arcs, and $P'_2$ is a $uv$-path whose final part is disjoint from $P'_1$ and only uses vertices in $L'_1, \dots, L'_{z-1}$ as internal vertices. Let $xy$ be the last backward arc of $P'_2$, then $P'_1$ and $P'_2[y,v]$ are disjoint. Now if $P'_1$ and $P'_2[u,y]$ intersect, we can extract from this closed walk a $sy$-path $P_{s,y}$ which is interally contained in $L'_1, \dots, L'_{z-1}$
Therefore, if $P_3$ denote any shortest $vt$-path in $D'$, then the union of $P_{s,y}$, $xy$, $P'_2[y,v]$ and $P_3$ is a $st$-path which uses at least one backward arc, and there it is a $st$-detour (TODO: figure). Note that this detour can have a different pair of breaking points, but this does not matter as we are enumerating over all possible pairs.

%Note that Abusing notation, all the detours mentionned in this section will concern detours with $(u,v)$ as breaking points. So in particular, when we say shortest detour, we mean shortest detour with $(u,v)$ as breaking points.

The algorithm will proceed in iterations, where at each iteration, it will maintain a set of vertices $F$ that can be used for the final part of $P_2$. More precisely, a subset $F$ of $L'_1, \dots, L'_{z-1}$ is said to be \textit{safe} if, for every $r \in F$, there exists two disjoint paths $P_{r,v}$ and $P_{s,u}$ using only forward arcs in $D'$ where $P_{r,v}$ is a $rv$-path and $P_{u,v}$ is a $su$-path. Moreover, we require that $P_{r,v}$ is contained in $F$. At the beginning, $F$ will simply be initialized as the set of all vertices in $L'_1, \dots, L'_{z-1}$ that satisfy these conditions. Because we are only using forward arcs, finding such paths can be done in polynomial time using the standard disjoint paths in DAGs algorithm. Moreover, note that by definition, the final part of $P_2$ is entirely contained in $F$. Throughout the algorithm, the set $F$ will be updated by removing vertices, while maintaining the property that the final part of $P_2$ is contained in $F$. 

Let us start with the following easy remark:

\begin{proposition}\label{pro:F}
   Suppose $F$ is a safe subset of vertices of $L'_1, \dots, L'_{z-1}$ and $x$ is an element of $F$ that can be reached from $u$ in $L'_1, \dots, L'_{z-1}$ via a path $H$ such that: 
   \begin{itemize}
    \item The internal vertices of $H$ are $F$-free
    \item The last arc of $H$ is a backward arc $yx$.
   \end{itemize}
   Then, denoting $P_{x,v}$ and $P_{s,u}$ the two paths certifying that $x \in F$, the $sv$-path obtained by simplifying $P_{s,u} \cdot H \cdot P_{x,v}$ is enterely contained in $(L'_1, \dots, L'_{z-1})$ and is longer than $d(s,v)$
\end{proposition}



At each iteration, the algorithm will start by computing the set of vertices $A_F$ that can be reached in $L'_1, \dots, L'_{z-1}$ from $u$ by paths avoiding $F$ for internal vertices and such that the last arc is a backward arc. This set can easily be computed with standard reachability algorithms. By proposition \ref{pro:F}, if $A_F \cap F$ is non-empty, then we can easily find a $st$-detour and the algorithm stops. Suppose now that $F \cap A_F$ is empty and remember that the algorithm invariant states that the final part of $P_2$ is contained in $F$. What this means is that the path $P_2$ cannot completely avoid $F$ before its final part and the algorithm will try to find elements of $F$ that can be identified as not belonging to this final part, which means it can safely be removed from $F$. This is the goal of the next lemma


\begin{lemma}
    There is a linear time algorithm which, assuming $F$ is a safe subset such that the final part of $P_2$ is contained in $F$, either finds a $st$-detour or outputs a subset $F' \subset F$ such that the final part of $P_2$ is contained in $F'$.
\end{lemma}

\begin{proof}
    Let $R$ denote the set of vertices of $L'_1, \dots, L'_{z-1}$ reachable in $D' \setminus F$ from $u$. The algorithm start by checking, if there exists a backward arc $xy$ from $R$ to $F$. If such an arc exsists, the algorithm then outputs the path obtained, using the same notation as Proposition \ref{pro:F}, by simplying $P_{s,u} \cdot H \cdot P_{x,v}$ and concatenating any shortest $vt$ path in $D$. By Proposition \ref{pro:F}, this is a $st$ detour and the algorithm terminates
    
    Suppose now that no such element exists and consider the first element $x$ of $P_2$ that does not belong to $R$. By definition of $R$, $x$ must then belong to $F$, and moreover, the arc $yx$ before $x$ on $P_2$ must be a forward arc. Let $F_R$ denote the set of elements of $F$ for which there exists a forward arc with tail in $R$. By the previous discussion $x$ must belong to $F_R$. Let $z \in L'_j$ be an element of $R_F$ that maximizes $j$, we claim that the last part of $P_2$ cannot use  
\end{proof}

then there exists a $su$-path $P_{s,u}$ and a $xv$-path $P_{x,v}$ using only forward arcs in $D'$ that are disjoint. Therefore, the union of $P_{s,u}$, $P_2[u,y]$, $yx$, and $P_{x,v}$ is a $st$-detour. The second task of the algorithm is to update the set $F$ by removing vertices while maintaining the following invariant:

, initialized as the union of $L'_1, \dots, L'_{z}$. It will also compute the set of vertices $I_{R} \subseteq V'$ such that for every $r \in I_R$, we can find two disjoint paths in $D'$: a $rv$-path $P_{r,v}$ and a $su$-path $P_{s,u} \subseteq R$ that uses forward arcs in $D'$. Intuitively, $I_R$ corresponds to the set of vertices that could potentially be in the last part of $P_2$. Note that because we are only using forward arcs, finding such paths can be done in polynomial time using the standard disjoint paths in DAGs algorithm. We will say that $I_R$ is the set \textit{associated} to $R$. Note that, using the previous notation, all the vertices of $P_{r,v}$ must be contained in $I_R$. Indeed, if $x \in P_{r,v}$, then $P_{r,v}[r,x]$ and $P_{s,u}$ are two disjoint paths using only forward arcs.

In each iteration, the algorithm will have two separates task. The first one is to find the pair of paths $P'_1$ and $P'_2$ with the following properties: 
\begin{itemize}
    \item $P'_1$ is a shortest $su$-path using only vertices of $R$ and forward arcs of $D'$.
    \item $P'_2$ is a $uv$-path whose final part is disjoint from $P'_1$ and only uses vertices in $I_R \cup \{v\}$. Moreover, $P'_2$ does not use any vertex of $R_V$ before its last backward arc.
    \item The length of $P'_2$ is minimized.
\end{itemize}
We call any pair of paths respecting the first two properties a \textit{valid pair} of paths for $V$. Note that by the previous discussion, having a valid pair gives rises to a $st$-detour of length at most $|P'_1| + |P'_2| + d(v,t)$. The following lemma basically follows from doing a shortest path algorithm in $(L'_1 \cup, L'_2, \dots, L'_{z} )\setminus I_R$. 

\begin{lemma}\label{lem:valid_pair}
    Given $R$ and its associated set $I_R$, finding a valid pair of paths $P'_1$ and $P'_2$ for $R$ that minimizes $|P'_2|$ can be done in polynomial time. 
\end{lemma}
The second task of the algorithm is to update the sets $V$ and remove vertices that cannot be used in $P_1$. This is the hard part of the algorithm and the goal will be to maintain the following invariant: 

\textbf{Invariant:} If the algorithm has not yet found a detour of length at most the shortest $st$-detour with $(u,v)$ as breaking points, then there exists a shortest $st$-detour $P$ with $(u,v)$ as breaking points such that $P_1$ is entirely contained in $R$.

To maintain the invariant throughout the algorithm, we will use the following lemma.

\begin{lemma}\label{lem:update_R}
    There exists an algorithm running in polynomial time such that the following properties are satisfied.
    Suppose that $R$ is a set of vertices such that some shortest $st$-detour $P$ with $(u,v)$ as breaking points has its final part of $P_2$ entirely contained in $R$. Moreover, suppose that no valid pair for $R$ gives rises to a shortest $st$-detour. Then the algorithm returns a set $R' \subset R$ such that $P_1$ is entirely contained in $R'$.
\end{lemma}

Using the algorithm of the previous lemmas, the overall algorithm for a pair $(u,v)$ works as follows. We initialize $R$ as the union of $L'_1, \dots, L'_{z}$ and repeat the following two steps until $R$ is empty.
\begin{itemize}
    \item Use the algorithm of Lemma \ref{lem:valid_pair} to find a valid pair of paths $P'_1$ and $P'_2$ for $R$ that minimizes $|P'_2|$.
    \item Update $R$ using the algorithm of Lemma \ref{lem:update_R}.
\end{itemize}
Finally, the algorithm returns the minimum length detour found during the execution.

\section*{Proof of Lemma \ref{lem:update_R}}
Let us now turn our attention to the proof of Lemma \ref{lem:update_R}. Suppose now we are given a set $R$ satisfying the condition of the lemma, meaning that there exists some shortest $st$-detour $P$ with $(u,v)$ as breaking points such that $P_1$ is entirely contained in $R$. Let $I_R$ be the set associated to $R$ and $B := \{u\} \cup ( L'_1 \cup L'_2 \cup \dots \cup L'_{z-1} )\setminus I_R$. Let us denote by $A$ the vertices of $B$ reachable from $u$ in $D'[B]$. Consider now $P_2$ the $uv$-path of $P$. By definition of $I_R$, the final part of $P_2$ must be contained in $I_R$. Let $x$ be the first vertex of $P_2 \cap I_R$ and let $y$ be the predecessor of $x$ in $P_2$. Since $u \in A$ and by definition of $A$, we have that $y \in A$. We will distinghuish two cases depending on whether $yx$ is a backward arc or a forward arc.

\begin{claim}
If $yx$ is a backward arc, then there exists a valid pair for $R$ that gives rises to a shortest $st$-detour.
\end{claim}

\begin{proof}
    Indeed, by definition of $R$, there exists two disjoint paths $P_{s,u}$ and $P_{x,v}$ in $D'$ that only uses forwards arcs. Moreover, we have that $P_{s,u}$ is entirely contained in $R$ and $P_{x,v}$ is entirely contained in $I_R$. Therefore, the pair $(P_{s,u}, P_2[u,y] \odot yx \odot P_{x,v})$ is a valid pair for $R$ that gives rises to a shortest $st$-detour. Indeed, $|P_{s,u}| = P_1$, and $|P_{x,v}| \leq P_2[x,v]$ by definition of $P_{x,v}$. Therefore, $|P_{s,u}| + |P_2[u,y]| + 1 + |P_{x,v}| \leq |P_1| + |P_2|$.
\end{proof}

Using the previous claim and the assumption that no valid pair for $R$ gives rise to a shortest $st$-detour, we deduce that $yx$ is a \textbf{forward} arc. Remember that we can always assume that every vertex of $R$ belongs to a $st$ path using only forward arcs. Let $X_A$ denote the set of vertices of $A$ that are the endpoints of a backward arc starting from $A$. By definition, we then have that $P_2$ contains a forward path in $D'$ from $X_A$ to $y$. The main goal of the lemma is then to understand the structures of $(X_A, I_R)$ forward paths in $D'$ to remove vertices from $R$ while maintaining the invariant. More precisely, how these $(X_A, I_R)$ forward paths can intersect the $su$ forward paths in $D'[R]$. Since we are only interested in forward paths, we can simply work with the DAG $D''$ obtained from $D'[R]$ by removing all backward arcs. Let $F_A$ denote the set of vertices of $R$ reachable from $X_A$ in $D'$ by a forward path which is internally disjoint from $R$. In particular, $F_A$ contains $X_A \cap R$. Let $I'_R$ denote the set of vertices of $R$ that can reach $I_R$ in $D'$ by a forward path which is internally disjoint from $R$. Note that $I'_R$ contains $I_R \cap R$ and by definition of $I_R$, any pair of $(s,u)$ path and $(F_A,I'_R)$ path in $D''$ must intersect. 


\begin{claim}\label{claim:grid}
    Noting $s := s_0$, there exists an integer $l \geq 1$ and a sequence of vertices $s_0, s_1, \dots, s_l$ of $D''$ such that for every $i \in [l-1]$, there is a path from $s_i$ to $s_{i+1}$ in $D''$. Moreover, denoting $D_i$ the set of vertices contained on such paths, we have that: 
    \begin{enumerate}
        \item For every $i \in [l-1]$, $D_i \cap F_A \neq \emptyset$ and $F_A \subseteq D_1 \cup D_1 \cup \dots \cup D_{l}$
        \item For every $i \in [l-1]$, $s_i$ is a vertex cup between $s$ and $F_A \setminus (D_1, \cup D_2, \dots, \cup D_{i-1})$
        \item For every $i \in [l-1]$, all $(D_i\cap F_A,I'_R)$-paths intersect $s_i$.
    \end{enumerate}
    Additionally, such a sequence can be found in polynomial time.
\end{claim}

\begin{proof}
    Consider $F_{A_0}$ the set of vertices of $F_A$ that are reachable from $s$ in $D''$ by a path which is internally disjoint from $F_A$. Since every $su$ path in $D''$ intersects every $(F_A,I'_R)$-path, we have that the maximum flow in $D''$ between $F_{A_0}$ and $\{u\} \cup I'_R$ is $1$. Indeed, suppose there exists a path $H_1$ from $x \in F_{A_0}$ to $u$ and a path $H_2$ from $y \in F_{A_0}$ to $I'_R$ that are disjoint. Since $x \in F_{A_0}$, there exists a path $H$ from $s$ to $x$ in $D''$ which is  disjoint from $y$ and therefore $H \odot H_1$ is a $(s,u)$-path in $D''$ which is disjoint from the $(F_A,I'_R)$ path $H_1$. Therefore, by Menger's theorem, there exists a vertex $s_1$ that separates $F_{A_0}$ from $\{u\} \cup I'_R$. Let $D_1$ be the set of vertices on a path from $s$ to $s_1$ in $D''$, we will show that $D_1$ and $s_1$ satisfy the last two properties of the claim. Suppose first that $a \in F_A \setminus D_1$ and consider a path $H$ from $s$ to $a$. By definition of $F_{A_0}$, we know that some vertex $x \in F_{A_0}$ must belong to that path. Moreover, by definition of $D''$, this path can be extended into a $su$ path, which means that $s_1$ must belong to that path, and since $a \not \in D_1$, it must appear before $s_1$ on that path. Therefore, $s_1$ is a vertex cut between $s$ and $F_A \setminus D_1$. Finally, the fact that $D_1$ is disjoint from $I'_R$ and every $(D_1,I'_R)$-path intersects $s_1$ follows from the definition of $s_1$.


    Assume now that the vertices $s_0, s_1, \dots s_j$ have been constructed for some $h \geq 1$ such that all the last two properties of the lemma hold for every $i \in [j]$ but $F_A \setminus (D_1, \cup D_2, \cdot, \cup D_{j})$ is not empty.  We will show how to construct $s_{j+1}$ in a similar way to $s_1$. Consider $F_{A_j}$ the set of vertices of $F_A$ that are reachable from $s_j$ in $D''$ by a path which is internally disjoint from $F_A$. Again, the maximum flow in $D''$ between $F_{A_j}$ and $\{u\} \cup I'_R$ is $1$. Indeed, suppose there exists a path $H_1$ from $x \in F_{A_j}$ to $u$ and a path $H_2$ from $y \in F_{A_j}$ to $I'_R$ that are disjoint. Since $x \in F_{A_j}$, there exists a path $H$ from $s_j$ to $x$ in $D''$ which is  disjoint from $y$ and therefore $H \odot H_2$ is a $(s_j,v)$-path in $D''$ which is disjoint from the $(F_A,v)$ path $H_1$. Moreover, since $D''$ is a DAG and $y$ is reachable from $s_j$, any $ss_j$ path is $D''$ must also be disjoint from $H_1$, which contradicts the fact that every $su$ path in $D''$ intersects every $(F_A,I'_R)$-path. Therefore, by Menger's theorem, there exists a vertex $s_{j+1}$ that separates $F_{A_j}$ from $\{u\} \cup I'_R$. Let $D_{j+1}$ be the set of vertices on a path from $s_j$ to $s_{j+1}$ in $D''$, we will show that $D_{j+1}$ and $s_{j+1}$ satisfy the last two properties of the claim. Again, the last property directly follows from the fact that $s_{j+1}$ is a $(F, I'_R)$ Suppose first that $a \in F_A \setminus (D_1, \cup D_2, \cup D_{j} \cup D_{j+1})$ and consider a path $H$ from $s$ to $a$. By definition of $F_{A_j}$, we know that some vertex $x \in F_{A_j}$ must belong to that path. Moreover, by definition of $D''$, this path can be extended into a $su$ path, which means that $s_{j+1}$ must belong to that path, and since $a \not \in (D_1, \cup D_2, \cup D_{j} \cup D_{j+1})$, it must appear before $s_{j+1}$ on that path. Therefore, $s_{j+1}$ is a vertex cut between $s$ and $F_A \setminus (D_1, \cup D_2, \cup D_{j})$. Finally, the fact that $D_{j+1}$ is disjoint from $I'_R$ and every $(D_{j+1},I'_R)$-path intersects $s_{j+1}$ follows from the definition of $s_{j+1}$.
\end{proof}





\end{document}